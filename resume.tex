
%%%%%%%%%%%%%%%%%%%%%%%%%%%%%%%%%%%%%%%%%
% "ModernCV" CV and Cover Letter
% LaTeX Template
% Version 1.1 (9/12/12)
%
% This template has been downloaded from:
% http://www.LaTeXTemplates.com
%
% Original author:
% Xavier Danaux (xdanaux@gmail.com)
%
% License:
% CC BY-NC-SA 3.0 (http://creativecommons.org/licenses/by-nc-sa/3.0/)
%
% Important note:
% This template requires the moderncv.cls and .sty files to be in the same 
% directory as this .tex file. These files provide the resume style and themes 
% used for structuring the document.
%
%%%%%%%%%%%%%%%%%%%%%%%%%%%%%%%%%%%%%%%%%

%----------------------------------------------------------------------------------------
%	PACKAGES AND OTHER DOCUMENT CONFIGURATIONS
%----------------------------------------------------------------------------------------

\documentclass[11pt,letterpaper,sans,aps]{moderncv} % Font sizes: 10, 11, or 12; paper sizes: a4paper, letterpaper, a5paper, legalpaper, executivepaper or landscape; font families: sans or roman

\moderncvstyle{classic} % CV theme - options include: 'casual' (default), 'classic', 'oldstyle' and 'banking'
\moderncvcolor{blue} % CV color - options include: 'blue' (default), 'orange', 'green', 'red', 'purple', 'grey' and 'black'
\usepackage{lipsum} % Used for inserting dummy 'Lorem ipsum' text into the template
\usepackage[scale=0.85]{geometry} % Reduce document margins
%\setlength{\hintscolumnwidth}{3cm} % Uncomment to change the width of the dates column
%\setlength{\makecvtitlenamewidth}{10cm} % For the 'classic' style, uncomment to adjust the width of the space allocated to your name

\newcommand{\myname}[1]{\underline{\textit{Francisco Machado}}}

%----------------------------------------------------------------------------------------
%	NAME AND CONTACT INFORMATION SECTION
%----------------------------------------------------------------------------------------

\firstname{Francisco} % Your first name
\familyname{Leal Machado} % Your last name

% All information in this block is optional, comment out any lines you don't need
\title{Curriculum Vitae}
\address{28 Fenway Drive}{Boston, MA 02215}
\mobile{(617) 682-9735}
\email{fmachado@mit.edu}
%\homepage{staff.org.edu/~jsmith}{staff.org.edu/$\sim$jsmith} % The first argument is the url for the clickable link, the second argument is the url displayed in the template - this allows special characters to be displayed such as the tilde in this example
%\extrainfo{additional information}
%\photo[70pt][0.4pt]{pictures/picture} % The first bracket is the picture height, the second is the thickness of the frame around the picture (0pt for no frame)
%\quote{"A witty and playful quotation" - John Smith}

%----------------------------------------------------------------------------------------

\begin{document}

\makecvtitle % Print the CV title

%----------------------------------------------------------------------------------------
%	EDUCATION SECTION
%----------------------------------------------------------------------------------------

\section{Education}


\cventry{2013--2016}{Massachusetts Institute of Technology}{\newline{}Candidate for Bachelor of Science Degree in Physics}{\newline{}Cambridge, MA}{\newline{}\textbf{GPA -- 5.0/5.0}}{}
\cventry{2012--2013}{Universidade de Coimbra}{\newline{}Candidate for Licence in Physics}{\newline{}Coimbra, Portugal}{\newline{}\textbf{GPA -- 5.0/5.0}}{} 



%----------------------------------------------------------------------------------------
%	WORK EXPERIENCE SECTION
%----------------------------------------------------------------------------------------

\section{Experience - Research}

\cventry{2015--Present}{Undergraduate Research}{\textsc{MIT Solid State Solar Thermal Energy Conversion (S$^3$TEC)}}{Cambridge}{Massachusetts}{Research in surface plasmon induced enhancement in electronic transitions
  \begin{itemize}
  \item Developing the code that will calculate the rates of the electronic transitions with different electromagnetic backgrounds.
  \item Designing different plasmon plasmon modes in order to selectively enhance particular transitions for an atom near the surface of our material.
  \end{itemize}
}

\cventry{2015}{Undergraduate Research}{\textsc{MIT Center for Material Science and Engineering}}{Cambridge}{Massachusetts}{Research in transport properties of electrons in low-dimensional materials.
\begin{itemize}
\item Fabricated devices which included selecting proper material flakes, charecterizating their properties and assembling in final device to be used in measurements.
\item Developed new device configurations to allow the better measurement of transport properties.
\end{itemize}}

\cventry{2014}{Undergraduate Research}{\textsc{MIT Kavli Institute for Astrophysics and Space Research}}{Cambridge}{Massachusetts}{Research in spectral data from galaxies from a simulation of the galaxy
\begin{itemize}
\item Analyzed how to make use of the simulated galaxies informations to better understand the properties of observable galaxies.
\item Developed tools that allow the matching between simulated and observed galaxies.
\end{itemize}}

\cventry{2014}{Undergraduate Research}{\textsc{MIT Aerospace Computational Design Lab}}{Cambridge}{Massachusetts}{Research in optimization of a numerical simulation of a stationary fluid flow
\begin{itemize} 
\item Analyzed and discovered the source of the major slow down in the program's run time.
\end{itemize}}


\cventry{2012--2013}{Undergraduate Research}{\textsc{Universidade de Coimbra - Physics Department}}{Coimbra}{Portugal}{Research in the topic of the dynamics of proteins and their protein reporter using computer simulations and stochastic models.
\begin{itemize}
\item Developed the simulation code used to run the simulations in the project.
\item Compiled and analyzed the data, presenting it at at a conferences
\item Presented results in poster format at the International Conference on Stem Cells for Drug Screening and Regenerative Medicine (2013)
%\item Currently working on broadening the search parameters of the simulation and implementing different protein behaviors in order to analyze different outcomes and later compare them with experimental data.
\end{itemize}}

\section{Experience - Work}
\cventry{2014}{Summer Intern}{\textsc{MemSql}}{San Francisco}{US}
{
\begin{itemize}
\item Worked directly on their C++ codebase
\item Developed and implemented features that were shipped to customers promptly.
\end{itemize}
}
\cventry{2013}{Senior Developer}{ \textsc{JeKnowledge}}{Coimbra}{Portugal}{}
\cventry{2012--2013}{Junior Developer}{\textsc{JeKnowledge}}{Coimbra}{Portugal}{
  Active Member of the Technology Department.
\begin{itemize}
\item Helped on the development of a human body detection software to analyze the correct movement of the body in various exercises.
\item Developed a glove prototype of a new product using Arduino technology.
\item Helped in the development of the data aquisition software for a new product in a start-up.
\end{itemize}
}


%------------------------------------------------------------------------------------
%	AWARDS SECTION
%------------------------------------------------------------------------------------


\section{Awards}

\cvitem{2015}{Winner of the Edward C. Pickering Award for the most Outstanding Original Project in the MIT Physics Junior Lab}
\cvitem{2013}{3\% Best Students Award at the University of Coimbra}
\cvitem{2013}{Bronze Medal at the ACM SouthWestern Regional Contest}
\cvitem{2012}{Bronze Medal at the International Physics Olympiads}
\cvitem{2012}{Bronze Medal at the International Olympiads of Informatics}
\cvitem{2012}{Gold Medal at the Portuguese University Programming Marathon}
\cvitem{2012}{Third Place in the Portuguese Olympiads of Informatics}
\cvitem{2011}{Honorable Mention at the IberoAmerican Mathematics Olympiads}
\cvitem{2011, 2012}{Silver Medal at the Portuguese Mathematics Olympiads}


%----------------------------------------------------------------------------------------
% Publications
%----------------------------------------------------------------------------------------

%\section{Publications}
\nocite{*}
\bibliographystyle{plain}
\bibliography{publications}


%----------------------------------------------------------------------------------------
% Conferences
%----------------------------------------------------------------------------------------

\section{Conferences}

\cventry{2013}{International Conference on Stem Cells for Drug Screening and Regenerative Medicine}{}{}{}
{
  \begin{itemize}
  \item Presented poster ``Following the Stochastic Dynamics of Nanog Through a Fluorescent Reporter - A Computational Study`` on work on DNA dynamics
  \end{itemize}
}


%-----------------------------------------------------------------------------------%
%      Summer Schools
%-----------------------------------------------------------------------------------%

\section{Summer Schools}

\cvitem{2015}{Novos Talentos Em Matem\'{a}tica Dynamical Systems Summer School}



%----------------------------------------------------------------------------------------
%	COMPUTER SKILLS SECTION
%----------------------------------------------------------------------------------------

%% \section{Skills}

%% \cvitem{Advanced}{\textsc{C, C++, Python}, Basic Algorithms,\textsc{\LaTeX, Emacs}}
%% \cvitem{Intermediate}{Data Analysis, Julia, Laboratory Work}
%% \cvitem{Basic}{\textsc{Java, Octave}}

%% \section{Relevant Classes}
%% \cvitemwithcomment{2014}{Junior Laboratory}{Experimental physics class}
%% \cvitemwithcomment{2014}{Introduction to Numerical Analysis}{}
%% \cvitemwithcomment{2013}{Quantum Mechanics and Quantum Computation - edX}{Introductory class}
%% \cvitemwithcomment{2013}{Differential Equations}{Solutions for ODE systems and introduction to nonlinear systems}
%% \cvitemwithcomment{2013}{Analysis of Algorithms}{Formal introduction to algorithms}



\section{Languages}
\cvitem{Portuguese}{Mothertongue}
\cvitem{English}{Fluent}
\cvitem{Spanish}{Basic}
\cvitem{French}{Basic}
\cvitem{German}{Basic}


%----------------------------------------------------------------------------------------

\end{document}

